%%%%%%%%%%%%%%%%%%%%%%%%%%%%%%%%%%%%%%%%%
% University/School Laboratory Report
% LaTeX Template
% Version 3.0 (4/2/13)
%
% This template has been downloaded from:
% http://www.LaTeXTemplates.com
%
% Original author:
% Linux and Unix Users Group at Virginia Tech Wiki 
% (https://vtluug.org/wiki/Example_LaTeX_chem_lab_report)
%
% License:
% CC BY-NC-SA 3.0 (http://creativecommons.org/licenses/by-nc-sa/3.0/)
%
%%%%%%%%%%%%%%%%%%%%%%%%%%%%%%%%%%%%%%%%%

%----------------------------------------------------------------------------------------
%	PACKAGES AND DOCUMENT CONFIGURATIONS
%----------------------------------------------------------------------------------------

\documentclass{article}

\usepackage{mhchem} % Package for chemical equation typesetting
\usepackage{siunitx} % Provides the \SI{}{} command for typesetting SI units

\usepackage{graphicx} % Required for the inclusion of images

\setlength\parindent{0pt} % Removes all indentation from paragraphs

\renewcommand{\labelenumi}{\alph{enumi}.} % Make numbering in the enumerate environment by letter rather than number (e.g. section 6)

%\usepackage{times} % Uncomment to use the Times New Roman font

%----------------------------------------------------------------------------------------
%	DOCUMENT INFORMATION
%----------------------------------------------------------------------------------------

\title{Mobile Application for Clinical Diagnoses  \\ UAP Proposal} % Title

\author{Ryan Lacey}

\date{}

\begin{document}

\maketitle % Insert the title, author and date

\begin{center}
\begin{tabular}{l r}
Advisor: & Dr. Rich Fletcher \\
Proposed: & March 7, 2014 \\
\end{tabular}
\end{center}

\section{Objective}

The purpose of this project is to develop a mobile application for use in clinics with limited resources in developing regions of India. The application will serve as a graphical interface through which clinicians can collect data with a low-cost spirometer and digital stethoscope. Ultimately the application will also process patient data to provide diagnostic suport for untrained healthworkers. \\



\section{Motivation}

Pulmonary disease (which includes asthma, COPD, pneumonia, lung cancer, tuberculosis) is an increasingly large portion of the global health challenges. Chronic Obstructive Pulmonary Disease (COPD) alone is currently the third leading cause of death in the world and second leading cause of death in India after ischemic heart disease (heart attacks); pneumonia is the leading cause of death in children under the age of 5. There is a great need to provide health workers in India a simple tool that can be used to screen for respiratory disease in the primary care setting and identify individuals that require clinical examination and intervention. The disease burden in India can be significantly reduced if there were better tools that could perform early detection of respiratory disease and enable clinical intervention (through medication or behavioral medicine) before the disease became more critical or developed further into chronic disease (COPD). For this project we are partnering with a top respiratory disease clinic in India.

\section{Goals}
\begin{enumerate}
\item[] \texttt{Phase I}\\ Develop graphical user interface for Android mobile devices.
\item[] \texttt{Phase II} \\ Integrate digital monitors into application to pass data through a low cost auxillary hub.
\item[] \texttt{Phase III} \\ Utilize machine learning algorithms to classify common disease states.
\end{enumerate}

The first phase is the most open ended portion of the project. Development of an intuitive user interface is always an iterative process and is likely to continue even as the project moves into the other phases. The goal this phase is to create a skeleton interface that will allow for basic functionality to be realized once the data connection is completed. This includes the main view of the application, any menus necessary for operation, and the first steps in graphical displays of patient data. The GUI will be developed in Java using the Android SDK foruse on phones and tablets. The ultimate goal is for the interface to be intuitive and language independant so that individuals with limitied technological experience will be able to utilize the application. \\

Phase II involves more software development, but at a lower level than Phase I. The  spirometer and digital stethoscope connect to the mobile device through XXXX. This connecting hub was developed in MIT's Development Lab with simplicity in mind so as to minimuze costs of production. Patient data will pass from the digital monitors through this hub into the application. \\

Following the completion of the hardware integration, time permitting, work will begin on a classificaion algorithm for common diseases. A simple machine learning algorithm will suffice to analyze patient data and return to the clinician most probable diseases the patient may have by comparing to a known set of classified patients. This algorithm should improve over time as the disease clinic in India sends over more classified patient data for the algorithm to train on. \\

\section{Requirements}



\end{document}